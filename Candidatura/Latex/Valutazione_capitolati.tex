\documentclass[a4paper,10pt]{article}

\usepackage[hidelinks]{hyperref}
\usepackage{float}

% Lingua
\usepackage[utf8]{inputenc}
\usepackage[italian]{babel}

% Margini
\usepackage{geometry}
\geometry{a4paper, top=3cm, bottom=3cm, left=3cm, right=3cm}
\setlength{\parindent}{0pt}

\usepackage{graphicx}   % per immagini
\usepackage{hyperref}   % per link
\usepackage{caption}    % per didascalie
\captionsetup[table]{labelformat=empty}
\usepackage{ifthen}     % per \ifthenelse

\usepackage{tikz}

\usepackage{array}      % per tabelle
\newcolumntype{M}[1]{>{\centering\arraybackslash}m{#1}}

\usepackage{datetime}

% Definisci un formato personalizzato
\newdateformat{italianformat}{\THEDAY~\monthname[\THEMONTH]~\THEYEAR}
\newdate{dataverbale}{29}{10}{2025}

% Definizione del formato con le barre
\newdateformat{slashdate}{\twodigit{\THEDAY}/\twodigit{\THEMONTH}/\THEYEAR}

% Per logo in ogni pagina
\usepackage{eso-pic}
\usepackage{tikz}
\AddToShipoutPictureBG{%
  \ifthenelse{\value{page}>1}{%
    \begin{tikzpicture}[remember picture, overlay]
      % Icona
      \node[anchor=north west, inner sep=0pt] at ([xshift=28pt,yshift=-28pt]current page.north west) 
        {\includegraphics[width=1cm, height=1cm]{resources/Tool-8_icon_white.png}};
      
      % Riga
      \draw[black, line width=0.5pt] ([xshift=28pt + 1cm, yshift=-28pt - 0.6cm]current page.north west) 
            -- ++(3.5cm, 0);
      % Testo
      \node[anchor=west, text=black, font=\small] at ([xshift=28pt + 1cm, yshift=-28pt - 0.4cm]current page.north west) 
        {Valutazione capitolati };
    \end{tikzpicture}
  }{}
}


%------Ridefinizione maketitle-------
\makeatletter
\renewcommand{\maketitle}{%
  \vspace*{\fill}
  \begin{center}
    {\LARGE \bfseries \@title \par}
    \vskip 1em
    {\large \@author \par}
    \vskip 1em
    {\large31/10/2025 \par}
  \end{center}
  \vspace*{\fill}
}
\makeatother
%-----------------------------------

\title{%
  \includegraphics[width=10cm]{resources/Tool-8_white_2.png} \\[1.5ex]
  \textbf{\LARGE Valutazione capitolati}
}
\author{}
\date{\dataverbale}
\begin{document}
\tikz[remember picture,overlay] \node[opacity=0.2,inner sep=0pt] at (current page.center){\includegraphics[width=\paperwidth,height=\paperheight]{resources/Abstract_lines.png}};

\maketitle

\newpage

\section*{Revisioni}
\begin{table}[h!]
\centering
\begin{tabular}{|M{1cm}|p{3cm}|p{7cm}|M{1.8cm}|}
  \hline
  \textbf{Ver.} & \textbf{Autore} & \textbf{Modifiche} & \textbf{Data} \\
  \hline
  1.0  & Niccolò Feltrin & Revisione e conferma documento & 30/10/2025 \\
  \hline
  0.1 & Stefano Maso & Prima stesura & \slashdate{\displaydate{dataverbale}} \\
  \hline
\end{tabular}
\label{tab:revisioni}
\end{table}

\newpage


\section*{Valutazione capitolati}
Nella riunione interna del 15/10/2025, il gruppo ha valutato ciascun capitolato e deciso di effettuare la candidatura per il progetto \textit{Second Brain} (C6). Di seguito, ecco le considerazioni su ciascun capitolato:
\begin{itemize}
    \item \textbf{C1, Automated EN18031 Compliance Verification}: ritenuto poco avvincente dalla maggior parte del gruppo;
    \item \textbf{C2, Code Guardian}: nonostante la grande disponibilità da parte dell'azienda, il progetto richiede tecnologie particolarmente complicate;
    \item \textbf{C3, DIPReader}: l'obiettivo che si pone il progetto ha catturato l'attenzione del gruppo; tuttavia, le tecnologie consigliate non sono familiari ai membri del gruppo;
    \item \textbf{C4, L'app che Protegge e Trasforma}: le tante funzionalità richieste, in particolare quelle sulla sicurezza dei dati, risultano particolarmente difficili da soddisfare;
    \item \textbf{C5, NEXUM}: progetto molto interessante, soprattutto per il problema che si propone di risolvere e anche per l'utilizzo dell'intelligenza artificiale;
    \item \textbf{C6, Second Brain}: la libertà nella scelta delle tecnologie e l'utilizzo di LLM del progetto hanno fin da subito catturato l'attenzione di tutto il gruppo, anche la velocità nelle risposte e la disponibilità da parte dell'azienda è stata molto apprezzata;
    \item \textbf{C7, Sistema di acquisizione dati da sensori}: molto interessante il contesto e l'obiettivo del progetto;
    \item \textbf{C8, Smart Order}: la richiesta di utilizzo di metodologie di Machine Learning avanzato per la gestione automatizzata di ordini di acquisto rende il progetto ambizioso e interessante;
    \item \textbf{C9, View4Life}: la possibilità di utilizzare i sensori dell'azienda e poterci effettivamente lavorare nella pratica attira sicuramente l'attenzione e l'interesse di cimentarsi in questo progetto; richiede un impegno sostanziale da parte di tutti i membri del gruppo;
\end{itemize}


\end{document}