\documentclass[a4paper,10pt]{article}

\usepackage[hidelinks]{hyperref}
\usepackage{float}

% Lingua
\usepackage[utf8]{inputenc}
\usepackage[italian]{babel}

% Margini
\usepackage{geometry}
\geometry{a4paper, top=3cm, bottom=3cm, left=3cm, right=3cm}
\setlength{\parindent}{0pt}

\usepackage{graphicx}   % per immagini
\usepackage{hyperref}   % per link
\usepackage{caption}    % per didascalie
\captionsetup[table]{labelformat=empty}
\usepackage{ifthen}     % per \ifthenelse

\usepackage{tikz}

\usepackage{array}      % per tabelle
\newcolumntype{M}[1]{>{\centering\arraybackslash}m{#1}}

\usepackage{datetime}

% Definisci un formato personalizzato
\newdateformat{italianformat}{\THEDAY~\monthname[\THEMONTH]~\THEYEAR}
\newdate{dataverbale}{29}{10}{2025}

% Definizione del formato con le barre
\newdateformat{slashdate}{\twodigit{\THEDAY}/\twodigit{\THEMONTH}/\THEYEAR}

% Per logo in ogni pagina
\usepackage{eso-pic}
\usepackage{tikz}
\AddToShipoutPictureBG{%
  \ifthenelse{\value{page}>1}{%
    \begin{tikzpicture}[remember picture, overlay]
      % Icona
      \node[anchor=north west, inner sep=0pt] at ([xshift=28pt,yshift=-28pt]current page.north west) 
        {\includegraphics[width=1cm, height=1cm]{resources/Tool-8_icon_white.png}};
      
      % Riga
      \draw[black, line width=0.5pt] ([xshift=28pt + 1cm, yshift=-28pt - 0.6cm]current page.north west) 
            -- ++(4cm, 0);
      % Testo
      \node[anchor=west, text=black, font=\small] at ([xshift=28pt + 1cm, yshift=-28pt - 0.4cm]current page.north west) 
        {Verbale \italianformat{\displaydate{dataverbale}}};
    \end{tikzpicture}
  }{}
}


%------Ridefinizione maketitle-------
\makeatletter
\renewcommand{\maketitle}{%
  \vspace*{\fill}
  \begin{center}
    {\LARGE \bfseries \@title \par}
    \vskip 1em
    {\large \@author \par}
    \vskip 1em
    {\large\slashdate{\displaydate{dataverbale}} \par}
  \end{center}
  \vspace*{\fill}
}
\makeatother
%-----------------------------------

\title{%
  \includegraphics[width=10cm]{resources/Tool-8_white_2.png} \\[1.5ex]
  \textbf{\LARGE Verbale riunione}
}
\author{}
\date{\dataverbale}
\begin{document}
\tikz[remember picture,overlay] \node[opacity=0.2,inner sep=0pt] at (current page.center){\includegraphics[width=\paperwidth,height=\paperheight]{resources/Abstract_lines.png}};

\maketitle

\newpage

\section*{Revisioni}
\begin{table}[h!]
\centering
\begin{tabular}{|M{1cm}|p{3cm}|p{7cm}|M{1.8cm}|}
  \hline
  \textbf{Ver.} & \textbf{Autore} & \textbf{Modifiche} & \textbf{Data} \\
  \hline
  0.1 & Stefano Maso & Prima stesura & \slashdate{\displaydate{dataverbale}} \\
  \hline
  0.2 & Besnik Murtezan & Prima revisione & \slashdate{\displaydate{dataverbale}} \\
  \hline
\end{tabular}
\label{tab:revisioni}
\end{table}

\newpage

\section*{Partecipanti}
\textbf{Orario della riunione}: 15:00 - 19:15
\\
\textbf{Sede / piattaforma}: Discord
\\
A questa riunione hanno partecipato:
\begin{table}[h!]
\centering
\begin{tabular}{|c|c|}
  \hline
  \textbf{Nome} & \textbf{Ruolo}\\
  \hline
  Giuliano Banchieri &  \\
  \hline
  Gabriele Disa &  \\
  \hline
  Stefano Maso &  \\
  \hline
  Besnik Murtezan &  \\
  \hline
  Ruben Spadiliero &  \\
  \hline
  Niccolò Feltrin &  \\
  \hline
\end{tabular}
\label{tab:partecipazione}
\end{table}

\section*{Motivazioni riunione}
La riunione è stata indetta per iniziare e terminare l'analisi dei costi, stimando le ore produttive richieste per persona seguendo il regolamento e iniziare la stesura del documento di candidatura.

\newpage

\section*{Verbale}
Abbiamo discusso dei seguenti punti:
\begin{itemize}
    \item Analisi dei costi
    \begin{itemize}
        \item Discussione riguardo alle ore da assegnare a ciascun ruolo sulla base delle indicazioni del regolamento, di esperienze personali e ragionamenti condivisi;
        \item Creazione foglio Excel di supporto 
    \end{itemize}
    \item Documento candidatura
    \begin{itemize}
        \item 	Predisposizione file relativo al documento di candidatura e  inizio redazione contenuti, in particolare la motivazione della nostra scelta
    \end{itemize}
\end{itemize}

\section*{Conclusioni}
Abbiamo portato a termine tutti gli obiettivi posti a inizio riunione, terminando l'analisi dei costi e iniziando con la stesura della candidatura

\section*{To-Do}
\begin{itemize}
    \par
    \item Terminare il documento di candidatura
\end{itemize}

\end{document}