\documentclass[a4paper,10pt]{article}

\usepackage[hidelinks]{hyperref}
\usepackage{float}

% Lingua
\usepackage[utf8]{inputenc}
\usepackage[italian]{babel}

% Margini
\usepackage{geometry}
\geometry{a4paper, top=3cm, bottom=3cm, left=3cm, right=3cm}
\setlength{\parindent}{0pt}

\usepackage{graphicx}   % per immagini
\usepackage{hyperref}   % per link
\usepackage{caption}    % per didascalie
\captionsetup[table]{labelformat=empty}
\usepackage{ifthen}     % per \ifthenelse

\usepackage{array}      % per tabelle
\newcolumntype{M}[1]{>{\centering\arraybackslash}m{#1}}

% Per logo in ogni pagina
\usepackage{eso-pic}
\usepackage{tikz}
\AddToShipoutPictureBG{%
  \ifthenelse{\value{page}>1}{%
    \begin{tikzpicture}[remember picture, overlay]
      % Icona
      \node[anchor=north west, inner sep=0pt] at ([xshift=28pt,yshift=-28pt]current page.north west) 
        {\includegraphics[width=1cm, height=1cm]{resources/Tool-8_icon_white.png}};
      
      % Riga
      \draw[black, line width=0.5pt] ([xshift=28pt + 1cm, yshift=-28pt - 0.6cm]current page.north west) 
            -- ++(4cm, 0);
      % Testo
      \node[anchor=west, text=black, font=\small] at ([xshift=28pt + 1cm, yshift=-28pt - 0.4cm]current page.north west) 
        {Verbale 15 ottobre 2025};
    \end{tikzpicture}
  }{}
}


%------Ridefinizione maketitle-------
\makeatletter
\renewcommand{\maketitle}{%
  \vspace*{\fill}
  \begin{center}
    {\LARGE \bfseries \@title \par}
    \vskip 1em
    {\large \@author \par}
    \vskip 1em
    {\@date \par}
  \end{center}
  \vspace*{\fill}
}
\makeatother
%-----------------------------------

\title{%
  \includegraphics[width=10cm]{resources/Tool-8_white.png} \\[1.5ex]
  \textbf{\LARGE Verbale riunione}
}
\author{}
\date{15/10/2025}
\begin{document}

\maketitle

\newpage

\section*{Revisioni}
\begin{table}[h!]
\centering
\begin{tabular}{|M{3cm}|p{8cm}|M{1.8cm}|}
  \hline
  \textbf{Autore} & \textbf{Modifiche} & \textbf{Data} \\
  \hline
  Giuliano Banchieri & Prima stesura & 15/10/2025 \\
  \hline
\end{tabular}
\caption{Storico revisioni}
\label{tab:revisioni}
\end{table}

\newpage

\section*{Motivazioni riunione}
Questa prima riunione è stata fatta per conoscerci, per condividere i nostri pensieri riguardanti i vari capitolati e per scegliere nome e logo del gruppo.

\section*{Partecipanti}
A questa riunione hanno partecipato:
\begin{table}[h!]
\centering
\begin{tabular}{|c|c|}
  \hline
  \textbf{Nome} & \textbf{Ruolo}\\
  \hline
  Giuliano Banchieri &  \\
  \hline
  Gabriele Disa &  \\
  \hline
  Stefano Maso &  \\
  \hline
  Besnik Murtezan &  \\
  \hline
  Ruben Spadiliero &  \\
  \hline
  Niccolò Feltrin &  \\
  \hline
\end{tabular}
\label{tab:partecipazione}
\end{table}

\section*{Verbale}
Dopo un confronto tra tutti i membri, il gruppo ha concordato il nome e il logo che ci rappresenteranno.

\vspace{1\baselineskip}

Abbiamo discusso delle nostre capacità e della fattibilità e apprezzamento dei vari capitolati per trovarne uno adatto a tutti e siamo giunti alle seguenti conclusioni:
\begin{itemize}
    \item  Capitolato favorito:
    \begin{itemize}
        \item C6: Second Brain
    \end{itemize} 
    \item Altri capitolati interessanti:
    \begin{itemize}
        \item C9: View4Life
        \item C3: DIPReader
    \end{itemize}
\end{itemize}

Questi sono stati i 3 capitolati più apprezzati dai membri del gruppo e si rimanda alla prossima riunione la discussione delle domande da porre per chiarire eventuali dubbi.

\section*{Conclusioni}
\begin{itemize}
    \item Raccogliere domande sui capitolati
    \par
    \item Mettersi in contatto con le aziende
\end{itemize}

\end{document}